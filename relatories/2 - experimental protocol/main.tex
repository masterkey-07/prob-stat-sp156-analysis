\documentclass[a4paper,12pt]{article}
\usepackage[utf8]{inputenc}
\usepackage[T1]{fontenc}
\usepackage[brazilian]{babel}
\usepackage{indentfirst}
\usepackage{abntex2cite}
\usepackage{graphicx}
\usepackage{amsmath}
\usepackage{setspace}
\usepackage{times}

\title{Protocolo Experimental - Análise e Predição das Solicitações do SP156}
\author{João Vinicius Farah Colombini 159501\\
Victor Jorge Carvalho Chaves 156740}
\date{11/08/2024}

\begin{document}

\maketitle

\tableofcontents
\newpage

\section{Domínio da Aplicação}

\subsection{Descrição da Base de Dados}

A base de dados são os relatórios de 2012 até 2024 das solicitações via o Portal de Atendimento da Prefeitura de São Paulo (SP156). Essas solicitações correspondem a reclamações, dúvidas e pedidos de ajuda referentes a uma variedade de assuntos da cidade, como violência, ruas desgastas, informações, etc.

Contudo, não há a conclusão se o problema foi resolvido ou não pelos respectivos orgãos responsáveis.

\subsubsection{Descrição das Solicitações de 2012 até 2014}

Para as solicitações de 2012 até 2014, há os seguintes valores:
\begin{enumerate}
    \item Data de abertura
    \item Bairro
    \item Distrito
    \item Órgão
    \item Canal de atendimento
    \item Assunto
    \item Especificação do assunto
    \item Status da solicitação
    \item Data do parecer
\end{enumerate}

\subsubsection{Descrição das Solicitações de 2015 até 2024}

Para as solicitações de 2015 até 2024, há os seguintes valores:
\begin{enumerate}
    \item Data de abertura
    \item Canal
    \item Tema
    \item Assunto
    \item Serviço
    \item Logradouro
    \item Numero
    \item CEP
    \item Setor
    \item Quadra
    \item Latitude
    \item Longitude
    \item Data do Parecer
    \item Status da solicitação
    \item Orgão
    \item Data
    \item Nível
    \item Prazo Atendimento
    \item Qualidade Atendimento
    \item Atendeu Solicitação
\end{enumerate}

\subsection{Entendimento da Base de Dados}

Com a necessidade do entendimento dos dados para o desenvolvimento do modelo, é proposto as seguintes atividades

\begin{enumerate}
    \item Listagem de todas as categorias
    \item Correlação entre os dados
    \item Correlação entre Problema X Região
    \item Correlação entre Problema X Mês
    \item Correlação entre Problema X Ano
    \item Verificar o crescimento/decrescimento de cada problema por Ano e Bairro
    \item Visualizar as frequências de ocorrências de cada categoria
\end{enumerate}

\section{Pré-Processamento}

Na base de dados, há múltiplos desafios a se resolver, valores nulos, múltiplos formatos de dado (texto, data, ruas, categorias). Esses desafios devem ser tratados para ser possível o desenvolvimento do modelo.

\subsection{Tratamento dos Dados Faltantes}
Com o tratamento de valores faltantes, será análisado o melhor conjunto de técnicas a se aplicar, como:

\begin{enumerate}
    \item Remover coluna.
    \item Preencher o valor nulo pela média.
    \item Preencher o valor nulo com um modelo semi-supervisionado.
    \item Preencher o valor a partir de outro dado.
\end{enumerate}

\subsection{Transformação e Criação dos Dados}

Com os múltiplos formatos de dados, é necessário o tratamento deles para melhor treinamento do modelo, aplicar transformações como:

\begin{enumerate}
    \item Extrair mais informações das datas (quantidade de dias, mês, ano, etc.)
    \item Converter valores de texto para numeros.
    \item Extrair informações dos Endereços e Localização (Bairro, Região).
\end{enumerate}

\subsection{Seleção das Colunas para o Treinamento}:

A partir da verificação de correlação entre as váriáveis, será análisado e decidido qual os melhores dados para serem usados no treinamento.

A partir dos resultados, será possível tomar as seguintes ações:
\begin{enumerate}
    \item Apagar as colunas com menor correlação.
    \item Criar dados a partir da correlação entre duas colunas.
\end{enumerate}



\section{Reconhecimento de Padrões}
\subsection{Objetivos}
O objetivo principal é o reconhecimento e previsão dos problemas da cidade de São Paulo.

Tendo em foco uma categorização mais objetiva de cada solicitação, como também a capacidade de previsão da origem e momento do problema.

\subsection{Algoritmos}

Nessa sessão será apresentado os algoritmos a serem testados e validados, cocm o intuito de encontrar o melhor para resolver os objetivos apresentados com acurácia e eficiência.

\subsubsection{Classificação}
\begin{itemize}
    \item Árvore de decisão para classificação
    \item Boosted trees para classificação
\end{itemize}

\subsubsection{Regressão}
\begin{itemize}
    \item Árvore de decisão para regressão
    \item Boosted Trees para regressão
    \item Regressão linear
    \item Random Forests
\end{itemize}

\subsubsection{Clusterização}
\begin{itemize}
    \item K-means
\end{itemize}
\subsection{Bibliotecas utilizadas}

\begin{enumerate}
    \item Análise e Processamento de Dados: Numpy, Pandas
    \item Bibliotecas de ML: Scikit-Learn, Pytorch, XGBoost
    \item Visualização: Matplotlib, Plotly
\end{enumerate}

\subsection{Metodos de validação}

Nessa sessão, será apresentada métodos de validação para validar a eficiência e acurácia dos modelos, em comparação a outros.

\subsubsection{Classificação}
\begin{itemize}
    \item Validação Cruzada
    \item Acurácia, precisão, recall, F1-Score
    \item Matriz de confusão
\end{itemize}
\subsubsection{Regressão}
\begin{itemize}
    \item Validação Cruzada
    \item MSE, RMSE, MAE, MAPE
\end{itemize}
\subsubsection{Clusterização}
\begin{itemize}
    \item Inertia e Silhouette Score
\end{itemize}

\section{Pós-Processamento}

Nessa sessão, será apresentado análise para demonstrar o padrão de comportamento por região, e alguns exemplos de predições.

\subsection{Análise dos Resultados}

A partir dos resultados dos modelos escolhidos, o objetivo é apresentar medidas de segurança e prevenção da cidade de São Paulo com base em predições e análises para cada região.

\subsection{Visualização das Informações}

\subsubsection{Entendimento dos Dados}

Utilizando uma série de métodos de visualização dos dados como gráficos de barra, box plot, mapa de temperatura, mapas, dentre outros. Com o objetivo de melhorar a interpretação e o domínio sobre os dados e sua natureza.

\subsubsection{Resultados dos Modelos}

Dependendo dos modelos escolhidos, apresentar os seus resultados de maneira que seja clara a sua validação em comparação a outros modelos.

Para cada tipo de modelo, será aplicado diferentes métodos de visualização para identificar e entender os melhores modelos e seus comportamentos durante os treinos.

\paragraph{Classificação}
\begin{itemize}
    \item Matriz de confusão (fica na forma de tabela)
    \item Curva de Precisão-Recall
    \item Grafico de dispersão
\end{itemize}

\paragraph{Regressão}
\begin{itemize}
    \item Curva de aprendizado
    \item Gráfico de dispersão com linha de regressão
\end{itemize}

\paragraph{Clusterização}
\begin{itemize}
    \item Mapa de calor
    \item Gráfico de Silhouette
    \item Gráfico de dispersão
\end{itemize}
\end{document}